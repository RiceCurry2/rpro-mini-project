\chapter{Implementation}

\section{Publisher}
The ROS publisher publishes messages onto the ROS network of nodes for any subscriber to read. In the code this is handled as described below.\\
\\
\begin{lstlisting}
void printLog()
    {
        rpro_mini_project::logOutput log;

        log.comm = logVar;

        mine_pub.publish(log);

        ros::spinOnce;
    }     
\end{lstlisting}
The $logOutput$ class of the $rpro\_mini\_project$ namespace is instantiated as \textit{log}. Then the contents of the class member \textit{comm} are set equal to the contents of the \textit{logVar} string variable.

Next \textit{log} is passed as a parameter when the $mine\_pub.pubish$ function is called. $mine\_pub$ is described below. Lastly the class \textit{spinOnce} of the namespace \textit{ROS} is called. This causes ROS to run its functions once thus advertising the existence of a new message on the topic and pubishing the message.\\
\\
\begin{lstlisting}
    mine_pub = nh.advertise<rpro_mini_project::logOutput>("miningLog",10);
\end{lstlisting}
In $mine\_pub$ we have nodehandle.anvertise which advertises new messages onto the ROS network of nodes. In this case the message $rpro\_mini\_project::logOutput$ are advertised onto the topic \textit{miningLog} at 10Hz.

\newpage

\section{Subscriber}
The ROS subscriber subscribes to topics on the ROS network of nodes and reads any advertised messages. In the code this is handled as described below.\\
\\
\begin{lstlisting}
void printLog(const rpro_mini_project::logOutput& logOut) {
        if (logOut.comm == "cls")
            {
                system("clear");
            }
        else
        {
            std::cout << logOut.comm << std::endl;
        }
    }
\end{lstlisting}
The printLog function has a constant reference to $rpro\_mini\_project::logOutput$ which is instantiated as \textit{logOut}. When a message of the \textit{logOutput}-class is published in the \textit{miningLog}-topic the contents are passed through the if-statements.

If the content of the class-member \textit{comm} equals "cls" the \textit{system}-function is called with the parameter "clear" which clears the terminal window.

If \textit{comm} contains anything else, the contents are printed in the console.\\
\\
\begin{lstlisting}
mine_sub = nh.subscribe("miningLog", 1000, &MiningOutput::printLog, this);
\end{lstlisting}
In $mine\_sub$ we have nodehandle.subscribe which subscribes to new messages - in this case on the \textit{miningLog}-topic. The second parameter is the message queue - how many messages will be kept in memory if our system cannot keep up with the flow of messages. Next is the pointer to the class-member who handles incoming messages.