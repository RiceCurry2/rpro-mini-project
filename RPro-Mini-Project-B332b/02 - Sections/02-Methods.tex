\chapter{Methods} \label{methods}

\iffalse
Description of necessary hardware and software components. The project should at least feature 2 ROS nodes that are in communication with each other. And it should implement at least two functions. Usage of actual robot hardware is optional in this mini-project.

I have split this into two chapters, "methods" and "implementation".
\fi

This mini-project makes use of a few tools in computer programming. The C++ programming language and ROS (Robot Operating System) one of a few programming languages, an open-source language, designed specially in programming for any robotic related environment.

\section{C++}

C++ is built on the structure of the C programming language, meaning that both C and C++ share similarities such as language syntax, code structure, compiling process, and capability to make code on hardware level environment, to mention a few.

One major difference between C and C++ is that in C++, object-oriented programming has been included. \cite{GforG}\\

\section{Robotic Operation System (ROS)}

Can be described as a meta-operating system, also called middleware. It acts like an independent system between the operating that runs on the computer/laptop one is working on and lower-level hardware \cite{ros_core_components}.

ROS is mainly designed to be compatible with C++ and Python \cite{RoboticsBackend}.\\
\\
The core elements of ROS are three things:
\begin{enumerate}
\setlength{\itemsep}{0.05\baselineskip}
    \item Communications Infrastructure
    \item Robot-Specific Features
    \item Tools
\end{enumerate}

Having a communication infrastructure means that the behaviour of ROS has a setup like a network in computing. In network computing, one has a router that is connected to the computers on the network and acts as the go-between between computer-to-computer. In ROS it is called nodes and these nodes are programs talking with each other in network behaviour and ROS functions as the master, i.e. the router.

These nodes have functionalities of listening, or in ROS terms, subscribing, for data from other nodes and publishing data out to the network that other nodes can subscribe to.\\

ROS can also record data that runs through its network so one can analyse the data and behaviour at a later time, also use these recordings and re-simulate the data.

The robot-specific feature libraries that help one generate results, or part of the results. These libraries can be mapping, localisation, pose estimation, and navigation amongst others.\\

%ROS is open source, so features get added to the lists continuously.\\

Tools that come with ROS can help while building up the program(s). Rviz can be used for simulation, testing out the behaviour of the robot and its sensors in a virtual environment, and rqt for looking at the network of nodes running when we run our program(s) and look at the communication between the nodes \cite{ros_core_components}.

\newpage

\subsection{ROS Publisher}

With ROS Publishing function we send messages across ROS. This is one of the simpler functions of communication in ROS, called Messages. With Messages, the messages are sent out on the network with no concern that anyone is listening, or receiving them \cite{ROSPubSub}. %More advanced communication functions are Services or Actionlib.

\begin{figure}[!ht]

\begin{lstlisting}
    ros::Publisher "object" = nh.advertise<"template">("topic", "queue");
\end{lstlisting}
\vspace{-10mm}
\caption{ROS Publisher example}
\end{figure}

\begin{itemize}
\setlength{\itemsep}{0.05\baselineskip}

    \item \textbf{"object"}     \\Name of the instance of the Publisher method.
    
    \item \textbf{"template"}   \\How the data inside the message object is structured (i.e. target message).
    
    \item \textbf{"topic"}      \\Topic name telling subscribers where to find this published data on the network of nodes handled by the master, i.e. the router.
    
    \item \textbf{"queue"}      \\The amount of how many messages are stored of those that haven't been used, if newer messages are generated, and the queue is full, the oldest messages will be discarded, and the newest added on top.
\end{itemize}

\subsection{ROS Subscriber}

With ROS Subscribing function we receive messages across ROS. Same with the Publisher, this is one of the simpler functions of communication in ROS, called Messages. Messages are received, and the subscriber, the receiver, does not need to report that it has received anything \cite{ROSPubSub}. %More advanced communication functions are Services or Actionlib.

\begin{figure}[!ht]
\begin{lstlisting}
    ros::Subscriber "object" = n.subscribe("topic", "queue", "function");
\end{lstlisting}
\vspace{-10mm}
\caption{ROS Subscriber example}
\end{figure}

\begin{itemize}
\setlength{\itemsep}{0.05\baselineskip}

    \item \textbf{"object"}     \\Name of the instance of the Subscriber method.
    
    \item \textbf{"topic"}      \\Topic name, telling the master which published data this subscriber wants to connect to and receive data from. 
    
    \item \textbf{"queue"}      \\The amount of how many messages are stored of those that haven't been used, if newer messages are generated, and the queue is full, the oldest messages will be discarded, and the newest added on top. 
    
    \item \textbf{"function"}   \\When a message is received, the message will be sent to a function or another place, for further processing.
    
\end{itemize}

\newpage

\subsection{ROS custom messages}
Custom messages are user defined messages with specific content. The messages consist of several fields (lines) each containing a type and a name. The data can be accessed using \textit{messagename.field}.\\
\\
The message is created by making a text file with the \textit{.msg} extension which is compiled using catkin. The .msg-file is compiled into a header containing a class where the members are the fields of the original .msg-file.\\
\\
The custom messages can contain a range of types such as various integer variants, bool, strings, and arrays. For a complete list refer to ROS Wiki.\\
\\
Source: ROS Wiki\cite{ROS_msg}

\section{String builder}
In order to transmit text AND variable data from one node to the other, it is necessary to create a string builder, since ROS custom messages will only accept std::string messages\cite{ROS_msg} and a std::ostringstream variable is needed to store multiple strings and variables together.

Practically we did this by creating an ostringstream variable called \textit{stringBuilder} and inserting strings and variables into this as needed. Subsequently, the string variable for publication (logVar) is set equal to the stringBuilders \textit{.str}-member function.\\
\\
\begin{figure}[!ht]
\begin{lstlisting}
    std::ostringstream stringBuilder;
    std::string logVar:
    
    stringbuilder << string << var << anotherString << anotherVar;
    logVar = stringBuilder.str();
\end{lstlisting}
\vspace{-10mm}
\caption{String builder example}
\end{figure}

\section{Clearing the string}
Regularly string and ostringstream can be cleared using their \textit{.clear}-member function. In ROS messages, however, there are no clear functions. This is solved by inserting the ASCII return carriage (\textit{\textbackslash r}) command at the end of each text line. This forces the publisher to write the next line of text at the beginning of the string contained in the message.

%\newpage

%\subsection{Clear screen through ROS}
%When building a text display using text from one node and printing it in another, it is sometimes useful to be able to clear the screen in the subscribing node.

%We solved this be creating an if-statement which reads \textit{"cls"} as a parameter that calls \textit{system("clear");}. This way we are able to send a simple string to the subscribing node telling it that we want the screen cleared.
%\begin{lstlisting}
%    if (logOut.comm == "cls")
%        {
%            system("clear");
%        }
%\end{lstlisting}